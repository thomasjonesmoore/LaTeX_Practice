\documentclass{article}
\usepackage{amsfonts,amssymb,amsmath,amsthm}
\usepackage[margin=1in]{geometry}

\title{CSCI 301, Winter 2018\\Math Exercises \# 3}
\author{Thomas Jones-Moore}
\date{Due date:  Tuesday, February 6, midnight.}

\begin{document}

\maketitle

Prove each of the following statements twice, using two different
methods.  Choose among direct, contrapositive, and contradiction.  You
are free to choose which two to use.  Explicitly note on each proof which
method you are using, and be explicit about the logical form of your
proposition, its constituent propositions, and any negations you
make.


\begin{enumerate}
\item If two integers have the same parity, then their sum is even.
  \begin{description}
    \item[First proof, method:] Contrapositive
    \begin{proof} Assume $x$ and $y$ have opposte parity, with $x$ being even and $y$ being odd. \\ There are integers $k$ and $j$, for which $x=2k$ and $y=2j+1$ by definition of even and odd numbers. 
    \begin{alignat*}{2}
    	x+y &= 2k+2j+1 \\
        &=2(k+j)+1
    \end{alignat*}
    $2(k+j)+1$ is an odd number by definition.\\
    Therefore, if two integers have different parity, then their sum is odd, proving that if two integers have the same parity then their sum is even. 
    \end{proof}
    
    \item[Second proof, method:] Direct
  \end{description}
  \begin{proof} Assume $x$ and $y$ have the same parity. \\ \\
  Case 1. $x$ and $y$ are even. \\
  \hspace*{6mm} There exists integers $a$ and $b$ for which $x=2a$ and $y=2b$ by definition of even numbers.
  \begin{alignat*}{2}
  x+y &= 2a+2b \\
  &= 2(a+b)
  \end{alignat*}
  By definition of a even number, $2(a+b)$ will always be even. \\
  Therefore, two integers of the even parity have an even sum. \\
  \\ Case 2. $x$ and $y$ are odd. \\
  \hspace*{6mm} There exists integers $a$ and $b$ for which $x=2a+1$ and $y=2b+1$ by definition of even numbers.
  \begin{alignat*}{2}
  x+y &= 2a+1+2b+1 \\
  &= 2(a+b)+2
  \end{alignat*}
  By definition of a odd number, $2(a+b)+2$ will always be even. \\
  Therefore, two integers of the odd parity will always be even.
  \end{proof}
  
\item Suppose $a\in\mathbb{Z}$.  If $a^2$ is not divisible by 4,
  then $a$ is odd.
  \begin{description}
    \item[First proof, method:] Contrapositive
    \begin{proof}
    Suppose $a\in\mathbb{Z}$ and even. By definition of an even integer, let some integer $n$ allow $a=2n$.
    \begin{alignat*}{2}
    a^2 &= (2n)^2 
    &= 4n^2
    \end{alignat*}
    Therefore, knowing $n^2\in\mathbb{Z}$, $a^2$ is divisible by 4.
    \end{proof}
    \item[Second proof, method:] Direct
    \begin{proof}
    	Suppose $a^2$ is not divisible by 4. \\
        Everything divisible by 4 is even. \\
        $4+4+...+n$ for any integer $n$ will always be even. \\
        
    
    	Therefore, $a$ is odd.
    \end{proof}
  \end{description}

\item Suppose $a,b\in\mathbb{Z}$.  If $4\mid (a^2+b^2)$, then $a$ and
  $b$ are not both odd.
  \begin{description}
    \item[First proof, method:] Contradiction
    \begin{proof} Suppose $a=2j+1$ and $b=2i+1$ for some integers $j,i$. \\
    \begin{alignat*}{2}
    a^2 &= (2j+1)^2 \\
    &= 4j^2+4j+1 \\
    b^2 &= (2j+1)^2 \\
    &= 4i^2+4i+1 \\
    &= a^2+b^2=4i^2+4i+4j^2+4j+2 \\
    \end{alignat*}
    We must now show this cannot be divided by 4
    \begin{alignat*}{2}
    \frac{4i^2+4i+4j^2+4j+2}{4} 
    = i^2 + i+j^2+j+\frac{1}{2}
    \end{alignat*}
    Knowing $i,j\in\mathbb{N}$, this cannot be resolved to an integer. \\
    Therefore, if $a,b$ are odd, $(a^2+b^2)$ cannot be divided by 4.
    
    \end{proof}
    \item[Second proof, method:] Contrapositive
    \begin{proof}
    Suppose $a$ and $b$ are odd. Assume for a integer $q$ and $w$: $a=2q+1$, $b=2w+1$. \\
    $a^2+b^2=4q^2+4q+1+4w^2+4w+1=4(q^2+q+w^2+w)+2$ \\ 
    Cannot have both $(a^2+b^2) \equiv 0 \mod 4$ \\ AND  \\$(a^2+b^2) \equiv 2 \mod 4$
    \end{proof}
  \end{description}

\item Suppose $a,b,c\in\mathbb{Z}$.  If $a^2+b^2=c^2$ then $a$ or
  $b$ is even.
  \begin{description}
    \item[First proof, method:] Contradiction
    \begin{proof}
    Suppose $a,b$ are odd integers, and $c$ is either odd or even. Assume for a integer $q$ and $w$: $a=2q+1$, $b=2w+1$.\\ Following, $(a^2+b^2)=4(q^2+q+w^2+w)+2$ \\
    Let $c=2y$ or $c=2y+1$ for integer $y$, and $c^2=4y^2$ or $4(y^2+y)+1$. \\
    In both cases of $c$, $a^2+b^2=c^2$ does not hold true.
    
    \end{proof}
    \item[Second proof, method:] Contrapositive
    \begin{proof}
    Suppose $a=2j+1$ and $b=2i+1$ for some integers $j,i$.
    \begin{alignat*}{2}
    a^2+b^2 &= 4(j^2+i^2)+4(j+i)+2 \\
    &= 2(2(j^2+i^2)+2(j+i)+1) \\
    &= 2(odd number)
    \end{alignat*}
    Therefore, either $a$ or $b$ has to be even.
    \end{proof}
  \end{description}
  
\end{enumerate}


\end{document}